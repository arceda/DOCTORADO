\documentclass[10pt]{beamer}
\usepackage[english]{babel}
\usepackage[utf8]{inputenc}
\usepackage[T1]{fontenc}
\usepackage{helvet}

%-------------------------------------------------------
% INFORMATION IN THE TITLE PAGE
%-------------------------------------------------------

\newcommand{\cstitle}{\textbf{Artificial Intelligence}}
\subtitle[]{Multiple Sequence Alignment using Particle Swarm Optimization}
\newcommand{\cscourseCode}{Artificial Intelligence}
\newcommand{\csauthor}{MSc. Vicente Machaca Arceda}
\institute[UNSA]{Universidad Nacional de San Agustín}
\newcommand{\csemail}{vmachacaa@unsa.edu.pe}
\newcommand{\instituteabr}{UNSA}
\newcommand{\nameUp}{}
\date{2021}
\title[\cscourseCode]{\cstitle}
\author{\csauthor}
%%%%%%%%%%%%%%%%%

%-------------------------------------------------------
% CHOOSE THE THEME
%-------------------------------------------------------
\def\mycmd{0} % CS THEME
\def\mycmd{1} % MYTHEME
%-------------------------------------------------------

\if\mycmd1
	\usetheme[]{Feather}
	\newcommand{\chref}[2]{	\href{#1}{{\usebeamercolor[bg]{Feather}#2}} }
\else
	\usepackage{csformat}
	\newcommand{\chref}[3][blue]{\href{#2}{\color{#1}{#3}}}%
\fi

\newcommand{\1}{
        	\setbeamertemplate{background}{
        		\includegraphics[width=\paperwidth,height=\paperheight]{img/1}
        		\tikz[overlay] \fill[fill opacity=0.75,fill=white] (0,0) rectangle (-\paperwidth,\paperheight);
        	}
}



%-------------------------------------------------------
% THE BODY OF THE PRESENTATION
%-------------------------------------------------------

\begin{document}


\AtBeginSubsection[]
{
    \begin{frame}
        \frametitle{Overview}
        \tableofcontents[currentsubsection]
    \end{frame}
}


%-------------------------------------------------------
% THE TITLEPAGE
%-------------------------------------------------------

\if\mycmd1 % MY THEME
	\1{
	\begin{frame}[plain,noframenumbering] 
		\titlepage 
	\end{frame}}

\else % CS THEME
	\maketitle
\fi


%-------------------------------------------------------
%-------------------------------------------------------
\begin{frame}{Content}
	\tableofcontents
\end{frame}
%-------------------------------------------------------
%-------------------------------------------------------


%%%%%%%%%%%%%%%%%%%%%%%%%%%%%%%%%%%%%%%%%%%%%%%%%%%%%%%%%%%%%%%%%%%%%%%%%%%%%%%%%%%%%%%%%%%%%%%%%%%%%%%%%%%%%%%%
%%%%%%%%%%%%%%%%%%%%%%%%%%%%%%%%%%%%%%%%%%%%%%%%%%%%%%%%%%%%%%%%%%%%%%%%%%%%%%%%%%%%%%%%%%%%%%%%%%%%%%%%%%%%%%%%
%%%%%%%%%%%%%%%%%%%%%%%%%%%%%%%%%%%%%%%%%%%%%%%%%%%%%%%%%%%%%%%%%%%%%%%%%%%%%%%%%%%%%%%%%%%%%%%%%%%%%%%%%%%%%%%%
\section{Introduction}
%%%%%%%%%%%%%%%%%%%%%%%%%%%%%%%%%%%%%%%%%%%%%%%%%%%%%%%%%%%%%%%%%%%%%%%%%%%%%%%%%%%%%%%%%%%%%%%%%%%%%%%%%%%%%%%%
%%%%%%%%%%%%%%%%%%%%%%%%%%%%%%%%%%%%%%%%%%%%%%%%%%%%%%%%%%%%%%%%%%%%%%%%%%%%%%%%%%%%%%%%%%%%%%%%%%%%%%%%%%%%%%%%
%%%%%%%%%%%%%%%%%%%%%%%%%%%%%%%%%%%%%%%%%%%%%%%%%%%%%%%%%%%%%%%%%%%%%%%%%%%%%%%%%%%%%%%%%%%%%%%%%%%%%%%%%%%%%%%%

%%%%%%%%%%%%%%%%%%%%%%%%%%%%%%%%%%%%%%%%%%%%%%%%%%%%%%%%%%%%%%%%%%%%%%%%%%%%%%%%%%%%%%%%%%%%%%%%%%%%%%%%%%%%%%%%
%%%%%%%%%%%%%%%%%%%%%%%%%%%%%%%%%%%%%%%%%%%%%%%%%%%%%%%%%%%%%%%%%%%%%%%%%%%%%%%%%%%%%%%%%%%%%%%%%%%%%%%%%%%%%%%%
%%%%%%%%%%%%%%%%%%%%%%%%%%%%%%%%%%%%%%%%%%%%%%%%%%%%%%%%%%%%%%%%%%%%%%%%%%%%%%%%%%%%%%%%%%%%%%%%%%%%%%%%%%%%%%%%
\subsection{Introduction}
%%%%%%%%%%%%%%%%%%%%%%%%%%%%%%%%%%%%%%%%%%%%%%%%%%%%%%%%%%%%%%%%%%%%%%%%%%%%%%%%%%%%%%%%%%%%%%%%%%%%%%%%%%%%%%%%
%%%%%%%%%%%%%%%%%%%%%%%%%%%%%%%%%%%%%%%%%%%%%%%%%%%%%%%%%%%%%%%%%%%%%%%%%%%%%%%%%%%%%%%%%%%%%%%%%%%%%%%%%%%%%%%%
%%%%%%%%%%%%%%%%%%%%%%%%%%%%%%%%%%%%%%%%%%%%%%%%%%%%%%%%%%%%%%%%%%%%%%%%%%%%%%%%%%%%%%%%%%%%%%%%%%%%%%%%%%%%%%%%


%%%%%%%%%%%%%%%%%%%%%%%%%%%%%%%%%%%%%%%%%%%%%%%%%%%%%%%%%%%%%%%%%%%%%%%%%%%%%%%%%%%%%%%%%%%%%%%%%%%%%%%%%%%%%%%%
%%%%%%%%%%%%%%%%%%%%%%%%%%%%%%%%%%%%%%%%%%%%%%%%%%%%%%%%%%%%%%%%%%%%%%%%%%%%%%%%%%%%%%%%%%%%%%%%%%%%%%%%%%%%%%%%
%%%%%%%%%%%%%%%%%%%%%%%%%%%%%%%%%%%%%%%%%%%%%%%%%%%%%%%%%%%%%%%%%%%%%%%%%%%%%%%%%%%%%%%%%%%%%%%%%%%%%%%%%%%%%%%%
\section{Papers}
%%%%%%%%%%%%%%%%%%%%%%%%%%%%%%%%%%%%%%%%%%%%%%%%%%%%%%%%%%%%%%%%%%%%%%%%%%%%%%%%%%%%%%%%%%%%%%%%%%%%%%%%%%%%%%%%
%%%%%%%%%%%%%%%%%%%%%%%%%%%%%%%%%%%%%%%%%%%%%%%%%%%%%%%%%%%%%%%%%%%%%%%%%%%%%%%%%%%%%%%%%%%%%%%%%%%%%%%%%%%%%%%%
%%%%%%%%%%%%%%%%%%%%%%%%%%%%%%%%%%%%%%%%%%%%%%%%%%%%%%%%%%%%%%%%%%%%%%%%%%%%%%%%%%%%%%%%%%%%%%%%%%%%%%%%%%%%%%%%


%%%%%%%%%%%%%%%%%%%%%%%%%%%%%%%%%%%%%%%%%%%%%%%%%%%%%%%%%%%%%%%%%%%%%%%%%%%%%%%%%%%%%%%%%%%%%%%%%%%%%%%%%%%%%%%%
%%%%%%%%%%%%%%%%%%%%%%%%%%%%%%%%%%%%%%%%%%%%%%%%%%%%%%%%%%%%%%%%%%%%%%%%%%%%%%%%%%%%%%
\subsection{Paper 1}
%%%%%%%%%%%%%%%%%%%%%%%%%%%%%%%%%%%%%%%%%%%%%%%%%%%%%%%%%%%%%%%%%%%%%%%%%%%%%%%%%%%%%%%%%%%%%%%%%%%%%%%%%%%%%%%%
%%%%%%%%%%%%%%%%%%%%%%%%%%%%%%%%%%%%%%%%%%%%%%%%%%%%%%%%%%%%%%%%%%%%%%%%%%%%%%%%%%%%%%

%-------------------------------------------------------
%-------------------------------------------------------
\begin{frame}{Paper 1}{}
	
	\begin{block}{}
		\centering
		\textbf{A Particle-based Method for Preserving Fluid Sheets} \cite{ando2011particle}.
	\end{block}

	\begin{itemize}
		\item \textbf{Year}: 2011
		\item \textbf{Authors}: Ando, Ryoichi and Tsuruno, Reiji
		\item \textbf{Event}: Proceedings of the 2011 ACM SIGGRAPH/Eurographics symposium on computer animation
	\end{itemize}
\end{frame}
%-------------------------------------------------------
%-------------------------------------------------------

%-------------------------------------------------------
%-------------------------------------------------------
\begin{frame}{Paper 1}{Problem and Proposal}
	\textbf{Problem:}	
	\begin{itemize}
		\item Particle based methods are no good for simulate thin fluids features.
	\end{itemize}
	
	\textbf{Proposal:}
	\begin{itemize}
		\item It is a particle-based framework that preserves thin fluid features like those in Eulerian fluid.
		\item Integrates Smoothed-Particle Hydrodynamics (SPH) and Particle-In-Cell/Fluid-Implicit-Particle (PIC/FLIP).
		\item The thin sheets are preserved by inserting new particles at sparse thin points in the sheets. These particles are then quickly removed as they
		dive into the deep water.
	\end{itemize}

\end{frame}
%-------------------------------------------------------
%-------------------------------------------------------


%-------------------------------------------------------
%-------------------------------------------------------
\begin{frame}[allowframebreaks]
	\frametitle{References}
	%\bibliographystyle{amsalpha}
	\bibliographystyle{IEEEtran}
	\bibliography{bibliography.bib}
\end{frame}
%-------------------------------------------------------
%-------------------------------------------------------


%-------------------------------------------------------
%-------------------------------------------------------
\if\mycmd1 % MY THEME
\1{
	{\1
		\begin{frame}[plain,noframenumbering]
			%\finalpage{Thank you}
			\begin{figure}[]
				\centering
				\includegraphics[width=\textwidth,height=0.7\textheight,keepaspectratio]{img/question.png}
				%\label{img:mot2}
				%\caption{Image example in 2 gray levels.}
			\end{figure}
	\end{frame}}
	\else % CS THEME
	\begin{frame}{Questions?}
		\begin{figure}[]
			\centering
			\includegraphics[width=\textwidth,height=0.7\textheight,keepaspectratio]{img/question.png}
			%\label{img:mot2}
			%\caption{Image example in 2 gray levels.}
		\end{figure}
		
	\end{frame}
	\fi
	%-------------------------------------------------------
	%-------------------------------------------------------
	

\end{document}